\documentclass[a4paper]{article}
\usepackage[utf8]{inputenc}
\usepackage[T1]{fontenc}
\usepackage[intlimits]{amsmath}
\usepackage{amsfonts}
\usepackage{amssymb}
\usepackage[export]{adjustbox}
\usepackage{graphicx}
\setlength{\parindent}{0pt}
\usepackage[left=1in, right=1in, top=1in, bottom=1in]{geometry}
\usepackage{float}
\usepackage{multicol}
\usepackage{listings}
\usepackage{xcolor}
\usepackage{cancel}
\usepackage{bm}
\usepackage{hyperref}
\setcounter{tocdepth}{1}
\usepackage[titletoc]{appendix}
\hypersetup{
	colorlinks=true,
	linkcolor=blue,
	filecolor=blue,
	urlcolor=blue,
}

\begin{document}
	
	\Huge\textbf{Motor Curve Generator}
	\newline
	\LARGE AMB Calculator
	
	\vspace{0.5cm}
	\normalsize
	
	This calculator generates motor curves in order to compare the speed, torque, current, power, and efficiency of various motors.
	
	\section*{Motor Equations}
	
	In FRC we describe DC motors using four parameters. The free speed, $ \omega_f $, is the maximum speed the motor can spin when not mechanically connected to anything. The stall torque, $ T_s $, is the maximum torque the motor can produce when not allowed to rotate. The free current, $ I_f $, is the minimum current the motor will draw (which occurs when spinning at max speed), and the stall current, $ I_s $, is the maximum current the motor will draw (which occurs when it is not allowed to rotate). These four values are measured when the motor is given a certain voltage, called the specification voltage $ V_{spec} $. In FRC this is almost always 12V. \\
	
	To find the motor speed $ \omega $ and current draw $ I $ at any applied torque $ T $, we use the following equations:
	
	\begin{equation} \label{w(T)}
		\omega (T) = \omega_f \left( 1 - \frac{T}{T_s} \right)
	\end{equation}
	
	\begin{equation} \label{I(T)}
		I (T) = \left( I_s - I_f \right) \cdot \frac{T}{T_s} + I_f
	\end{equation}
	\\
	The mechanical power generated by the motor $ P $ is equal to the output speed multiplied by the output torque, $ P(T) = T \cdot \omega (T) $. Plugging in the function $ \omega (T) $, we get:
	
	\begin{equation}
		P (T) = T \cdot \omega_f \left( 1 - \frac{T}{T_s} \right)
	\end{equation}
	\\
	The motor efficiency, $ \eta $, is the ratio of the output power to input power. The input power, in the form of electricity, is the voltage multiplied by the current drawn. So:
	
	\begin{equation}
		\eta (T) = \frac{P(T)}{V \cdot I (T)}
		= \frac{T \cdot \omega_f \left( 1 - \frac{T}{T_s} \right)}{V \cdot \left[ \left( I_s - I_f \right) \cdot \frac{T}{T_s} + I_f \right]}
		= \frac{T \cdot \omega_f \left( T_s - T \right)}{V \left[ T \left( I_s - I_f \right) + T_s I_f \right]}
	\end{equation}
	
	
	
	\section*{Motor Systems}
	
	We can combine multiple identical motors that are mechanically tied together into a motor system, with parameters derived from those of the individual motors. We will use $ \widetilde{\square} $ to denote the property of the motor system which contains $ n $ motors. All of the parameters scale linearly with the voltage applied to the motor, so we can adjust the parameters for the applied voltage $ V $ as well. Additionally, we can apply a gear reduction of $ G $ : 1 to the system, which will increase the motor torque at the expense of motor speed. \\
	
	The speed of the motor system is identical to the speed of each individual motor, divided by the gear ratio:
	
	\begin{equation}
		\widetilde{\omega}_f = \omega_f \cdot \frac{1}{G} \cdot \left( \frac{V}{V_{spec}} \right)
	\end{equation}
	\\
	The torque produced by the motor system is equal to the sum of the torques produced by each of the motors, multiplied by our gear ratio. Since all of the motors are identical:
	
	\begin{equation}
		\widetilde{T}_s = T_s \cdot n G \left( \frac{V}{V_{spec}} \right)
	\end{equation}
	\\
	Similarly, the total current drawn by the system is equal to the sum of the individual motor currents:
	
	\begin{equation}
		\widetilde{I}_f = I_f \cdot n \left( \frac{V}{V_{spec}} \right)\ ; \qquad\qquad \widetilde{I}_s = I_s \cdot n \left( \frac{V}{V_{spec}} \right)
	\end{equation}
	\\
	Therefore we can adjust the above motor equations to work with motor systems:
	
	\begin{equation}
		\widetilde{\omega} (T) = \widetilde{\omega}_f \left( 1 - \frac{T}{\widetilde{T}_s} \right)
		= \omega_f \cdot \frac{1}{G} \left( \frac{V}{V_{spec}} - \frac{T}{T_s \cdot n G} \right)
	\end{equation}
	
	\begin{equation}
		\widetilde{I} (T) = \left( \widetilde{I}_s - \widetilde{I}_f \right) \cdot \frac{T}{\widetilde{T}_s} + \widetilde{I}_f
		= \left( I_s - I_f \right) \cdot \frac{T}{T_s \cdot G} + I_f \cdot n \left( \frac{V}{V_{spec}} \right)
	\end{equation}
	
	\begin{equation}
		\widetilde{P} (T) = T \cdot \widetilde{\omega} (T)
		= T \cdot \omega_f \cdot \frac{1}{G} \left( \frac{V}{V_{spec}} - \frac{T}{T_s \cdot n G} \right)
	\end{equation}
	
	\begin{equation}
		\widetilde{\eta} (T) = \frac{\widetilde{P}(T)}{V \cdot \widetilde{I} (T)}
		= \frac{T \cdot \omega_f \left( T_s \cdot n G \left( \frac{V}{V_{spec}} \right) - T \right)}{V \cdot n G \left[ T \left( I_s - I_f \right) + I_f \cdot T_s \cdot n G \left( \frac{V}{V_{spec}} \right) \right]}
	\end{equation}


	\section*{Current Limit}
	
	Applying limits on the motor current is a common method for protecting the motor from overheating and the breakers from tripping. Since the current and torque are linearly related, this effectively limits the maximum torque the motor can provide. The new maximum torque $ T_{max} $ can be calculated by solving the current equation to find the torque at which the current limit $ I_{max} $ is reached:
	
	\begin{equation}
		T_{max} = \widetilde{T}_s \frac{I_{max} - \widetilde{I}_f}{\widetilde{I}_s - \widetilde{I}_f}
		= T_s \cdot G \cdot \frac{I_{max} - I_f \cdot n \left( \frac{V}{V_{spec}} \right)}{I_s - I_f}
	\end{equation}\\
	
	The domains of all four graphs are limited to $ T \in [0, T_{max}] $
	
	
	
\end{document}