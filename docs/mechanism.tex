\documentclass[a4paper]{article}
\usepackage[utf8]{inputenc}
\usepackage[T1]{fontenc}
\usepackage[intlimits]{amsmath}
\usepackage{amsfonts}
\usepackage{amssymb}
\usepackage[export]{adjustbox}
\usepackage{graphicx}
\setlength{\parindent}{0pt}
\usepackage[left=1in, right=1in, top=1in, bottom=1in]{geometry}
\usepackage{float}
\usepackage{multicol}
\usepackage{listings}
\usepackage{xcolor}
\usepackage{cancel}
\usepackage{bm}
\usepackage{hyperref}
\setcounter{tocdepth}{1}
\usepackage[titletoc]{appendix}
\hypersetup{
	colorlinks=true,
	linkcolor=blue,
	filecolor=blue,
	urlcolor=blue,
}

\begin{document}
	
	\Huge\textbf{Mechanism Calculator}
	\newline
	\LARGE AMB Calculator
	
	\vspace{0.5cm}
	\normalsize
	
	This calculator is used to find the properties of a mechanism given its gear ratio, or to find the proper gear ratio needed to give the mechanism certain properties. A mechanism is defined as a system powered by one or more motors with a constant gear ratio connecting the motor to a constant load. 
	
	\section*{Motor Equations}
	
	In FRC we describe DC motors using four parameters. The free speed, $ \omega_f $, is the maximum speed the motor can spin when not mechanically connected to anything. The stall torque, $ T_s $, is the maximum torque the motor can produce when not allowed to rotate. The free current, $ I_f $, is the minimum current the motor will draw (which occurs when spinning at max speed), and the stall current, $ I_s $, is the maximum current the motor will draw (which occurs when it is not allowed to rotate). These four values are measured when the motor is given a certain voltage, called the specification voltage $ V_{spec} $. In FRC this is almost always 12V. \\
	
	To find the motor speed $ \omega $ and current draw $ I $ at any applied torque $ T $, we use the following equations:
	
	\begin{equation} \label{w(T)}
		\omega (T) = \omega_f \left( 1 - \frac{T}{T_s} \right)
	\end{equation}
	
	\begin{equation} \label{I(T)}
		I (T) = \left[ \left( I_s - I_f \right) \cdot \frac{T}{T_s} + I_f \right]
	\end{equation}
	\\
	The mechanical power generated by the motor $ P $ is equal to the output speed multiplied by the output torque, $ P(T) = T \cdot \omega (T) $. Plugging in the function $ \omega (T) $, we see that the maximum power $ P_{max} $ occurs at $ T = \frac{1}{2} T_s $, or:
	
	\begin{equation}
		P_{max} = \Big[ T \cdot \omega(T) \Big]_{T = \frac{1}{2} T_s} = \frac{T_s\, \omega_f}{4}
	\end{equation}
	\\
	The motor efficiency, $ \eta_{motor} $, is the ratio of the output power to input power. The input power, in the form of electricity, is the voltage multiplied by the current drawn. So:
	
	\begin{equation}
		\eta_{motor} (T) = \frac{P(T)}{V \cdot I (T)}
		= \frac{T \cdot \omega_f \left( 1 - \frac{T}{T_s} \right)}{V \cdot \left[ \left( I_s - I_f \right) \cdot \frac{T}{T_s} + I_f \right]}
		= \frac{T \cdot \omega_f \left( T_s - T \right)}{V \left[ T \left( I_s - I_f \right) + T_s I_f \right]}
	\end{equation}
	
	
	\section*{Motor Systems}
	
	We can combine multiple identical motors that are mechanically tied together into a motor system, with parameters derived from those of the individual motors. We will use $ \widetilde{\square} $ to denote the property of the motor system which contains $ n $ motors. When combining the motors in a gearbox, there is an imperfect power transmission; this efficiency percentage will be denoted as $ \eta $. All of the parameters scale linearly with the voltage applied to the motor, so we can adjust the parameters for the applied voltage $ V $ as well. \\
	
	The speed of the motor system is identical to the speed of each individual motor:
	
	\begin{equation}
		\widetilde{\omega}_f = \omega_f \left( \frac{V}{V_{spec}} \right)
	\end{equation}
	\\
	The torque produced by the motor system is equal to the sum of the torques produced by each of the motors, multiplied by our gearbox efficiency fraction. Since all of the motors are identical:
	
	\begin{equation}
		\widetilde{T}_s = T_s \cdot n \eta \left( \frac{V}{V_{spec}} \right)
	\end{equation}
	\\
	Similarly, the total current drawn by the system is equal to the sum of the individual motor currents:
	
	\begin{equation}
		\widetilde{I}_f = I_f \cdot n \left( \frac{V}{V_{spec}} \right)\ ; \qquad\qquad \widetilde{I}_s = I_s \cdot n \left( \frac{V}{V_{spec}} \right)
	\end{equation}
	\\
	And plugging these into the equation for the motor efficiency:
	
	\begin{equation}
		\widetilde{\eta}_{motor} = \frac{T \cdot \widetilde{\omega}_f \left( \widetilde{T}_s - T \right)}{V \left[ T \left( \widetilde{I}_s - \widetilde{I}_f \right) + \widetilde{T}_s \widetilde{I}_f \right]}
	\end{equation}
	
	
	\section*{Mechanism Definition}
	
	We define a mechanism as any system powered by a motor or system of identical motors with a constant gear ratio $ G $, which can have either a rotational or linear output. All mechanisms are loaded with a constant force $ F $ at a constant radius $ r $, which depend on the specific geometry and usage. For linear mechanisms, the load radius is the radius of the wheel, pulley, sprocket, etc. that translates between rotational and linear movement. For rotational mechanisms, the it is the perpendicular distance between the force and the axis of rotation.\\
	
	We will express the output speed in two ways. The free speed is the mechanism's output speed when no load is connected. The loaded speed is the output speed when the load force is applied. These speeds can be expressed as either linear velocities, $ v_{free} $ and $ v_{load} $, or rotational velocities, $ \omega_{free} $ and $ \omega_{load} $. In all cases, the speed in question is a steady-state speed, meaning the mechanism has been powered on for enough time to reach it's final speed. It does not take into account the acceleration of the mechanism while in start-up or as the load changes.\\
	
	Since one common limitation is the breaker on each motor, we will calculate the per-motor current $ I $ when the given load is applied. In addition, we define the stall load, $ F_s $, which is the force needed to get the loaded speed to zero (i.e. the mechanism cannot move). We also define the stall voltage, $ V_s $, as the voltage that, when applied to the motor(s), would cause the loaded speed to fall to zero under the defined force. This is useful for ensuring the voltage to hold a certain force for a prolonged time is low enough to be safe for the motor.\\
	
	These quantities are related to each other through the following equations:
	
	\begin{align}
		\omega_{free} = \frac{\widetilde{\omega}_f}{G}\ &; \qquad v_{free} = 2\pi r \cdot \omega_{free} \label{wfree} \\
		\omega_{load} = \omega_{free} \left( 1 - \frac{F r}{\widetilde{T}_s G} \right)\ &; \qquad v_{load} = 2\pi r \cdot \omega_{load} \label{wload}
	\end{align}
	\begin{equation} \label{current}
		I = \frac{1}{n} \left[ \frac{F r}{\widetilde{T}_s G} \left( \widetilde{I}_s - \widetilde{I}_f \right) + \widetilde{I}_f \right]
	\end{equation}
	\begin{equation} \label{Fs}
		F_s = \frac{\widetilde{T}_s G}{r}
	\end{equation}
	\begin{equation} \label{Vs}
		V_s = \frac{F r}{T_s n \eta G} \cdot V_{spec}
	\end{equation}
	
	\vspace{3mm}
	\section*{Gear Ratio Calculation}
	
	In order to set certain characteristics of the mechanism, we will derive equations to find the gear ratio when given each characteristic.\\
	
	\newpage
	For the free rotational and linear velocities, we solve for $ G $ to get:
	
	\begin{equation}
		G = \frac{\widetilde{\omega}_f}{\omega_{free}} \qquad\qquad G = \frac{2\pi r \widetilde{\omega}_f}{\omega_{free}}
	\end{equation}
	\\
	For the loaded rotational velocity, we will substitute (\ref{wfree}) into (\ref{wload}) and solve for $ G $:
	
	\begin{equation}
		\omega_{load} = \frac{\widetilde{\omega}_f}{G} \left( 1 - \frac{F r}{\widetilde{T}_s G} \right) \implies
		G = \frac{\widetilde{\omega}_f}{2 \omega_{load}} \left( 1 + \sqrt{1 - 4 \frac{F r}{\widetilde{T}_s} \cdot \frac{\omega_{load}}{\widetilde{\omega}_f}} \right)
	\end{equation}
	\\
	And for the loaded linear velocity we substitute in $ v_{load} = 2\pi r \cdot \omega_{load} $ :
	
	\begin{equation}
		G = \frac{\pi r \widetilde{\omega}_f}{v_{load}} \left( 1 + \sqrt{1 - 4 \frac{F}{\widetilde{T}_s} \cdot \frac{v_{load}}{2\pi \widetilde{\omega}_f}} \right)
	\end{equation}
	\\
	For the per-motor current, we solve (\ref{current}) for $ G $ to get:
	
	\begin{equation}
		G = \frac{\widetilde{T}_s}{F r} \left( \frac{I n - \widetilde{I}_f}{\widetilde{I_s} - \widetilde{I}_f} \right)
	\end{equation}
	\\
	For stall load, we solve (\ref{Fs}) for $ G $:
	
	\begin{equation} \label{Fs_G}
		G = \frac{F_s r}{\widetilde{T}_s}
	\end{equation}
	\\
	And for stall voltage we solve (\ref{Vs}) for $ G $:
	
	\begin{equation}
		G = \frac{F r}{T_s n \eta \left( \frac{V_s}{V_{spec}} \right) }
	\end{equation}
	\\
	Finally we define three important characteristic points: Max Power, Max Efficiency, and Stall. At stall, the stall load is equal to the applied load (i.e. $ F_s = F $). Substituting into (\ref{Fs_G}):
	
	\begin{equation}
		G = \frac{F r}{\widetilde{T}_s}
	\end{equation}
	\\
	We know that maximum power for a single motor occurs at half of the maximum load. Expanding this to an entire mechanism, maximum power occurs when $ F = \frac{1}{2} F_s $. Inserting this into (\ref{Fs_G}) gives:
	
	\begin{equation}
		G = \frac{2 F r}{\widetilde{T}_s}
	\end{equation}
	\\
	In order to find the maximum efficiency, we will take the derivative of the efficiency equation and set it equal to zero:
	
	\begin{equation}
		0 = \frac{d \eta_{motor}}{dT} = \frac{\widetilde{\omega}_f \left[ \widetilde{I}_f (T - \widetilde{T}_s)^2 - \widetilde{I}_s T^2 \right]}{V \left[ \widetilde{I}_s T + \widetilde{I}_f (\widetilde{T}_s - T) \right]^2}
	\end{equation}
	\\
	Solving for the value of $ T $ at which $ \eta_{motor} $ is maximized:
	
	\begin{equation}
		T_{max} = \widetilde{T}_s \frac{\sqrt{\widetilde{I}_f}}{\sqrt{\widetilde{I}_s} + \sqrt{\widetilde{I}_f}}
	\end{equation}
	\\
	We know that $ T_{manip} = G \cdot T_{motor} $, so we can substitute $ T_{max} $ for $ T_{motor} $:
	
	\begin{equation}
		G = \frac{T_{manip}}{T_{motor}} = \frac{F r}{T_{max}}
		= \frac{F r}{\widetilde{T}_s \frac{\sqrt{\widetilde{I}_f}}{\sqrt{\widetilde{I}_s} + \sqrt{\widetilde{I}_f}}}
		= \frac{F r}{\widetilde{T}_s} \left( 1 + \sqrt{\frac{\widetilde{I}_s}{\widetilde{I}_f}} \right)
	\end{equation}
	
	
	
	
	
\end{document}